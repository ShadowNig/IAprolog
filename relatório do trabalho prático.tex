\documentclass[12pt,a4paper]{article}
\usepackage[utf8]{inputenc}
\usepackage[portuguese]{babel}
\usepackage[T1]{fontenc}
\usepackage{amsmath}
\usepackage{amsfonts}
\usepackage{amssymb}
\usepackage{graphicx}
\usepackage[left=2cm,right=2cm,top=2cm,bottom=2cm]{geometry}
\author{Gabriel de Albuquerque Mendes, João Pedro Abreu de Souza e Maqueise Pinheiro}
\title{Um titulo maneiro entra aqui}
\date{14 de Novembro de 2019}
\begin{document}
\maketitle
O \textit{8-Puzzle} (ou quebra-cabeça de 8 peças) é um quebra-cabeça composto por uma caixa oca com 8 peças deslizantes dentro, todas numeradas de 1 à 8. O objetivo do jogo é arrumar as peças de forma que os números fiquem em ordem crescente e um espaço no final.

Nosso trabalho visa resolver o 8-Puzzle utilizando algoritmos de busca implementados em Prolog.
\\

Inicialmente, foi feito um algoritmo utilizando busca em profundidade. Essa abordagem não se mostrou eficiente, já que percorria um longo caminho até o resultado esperado.

Em seguida, foi construido um algoritmo usando busca em largura. Esse método de busca se mostrou mais eficaz que o anterior, com caminhos mais diretos até o resultado.

...

\end{document}