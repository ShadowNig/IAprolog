\documentclass[12pt,a4paper]{article}
\usepackage[utf8]{inputenc}
\usepackage[portuguese]{babel}
\usepackage[T1]{fontenc}
\usepackage{amsmath}
\usepackage{amsfonts}
\usepackage{amssymb}
\usepackage{graphicx}
\usepackage[left=2cm,right=2cm,top=2cm,bottom=2cm]{geometry}
\author{Gabriel de Albuquerque Mendes,\\
 João Pedro Abreu de Souza e \\
 Maqueise de Medeiros Pinheiro}
\title{Resolvendo o 8-Puzzle com algoritmos de busca em Prolog}
\date{19 de Novembro de 2019}

\begin{document}
\maketitle
O \textit{8-Puzzle} (ou quebra-cabeça de 8 peças) é um quebra-cabeça composto por uma caixa oca com 8 peças deslizantes dentro, todas numeradas de 1 à 8. O objetivo do jogo é arrumar as peças de forma que os números fiquem em ordem crescente e um espaço no final.

Nosso trabalho visa resolver o 8-Puzzle utilizando algoritmos de busca implementados em Prolog.
\\

Inicialmente, foi feito um algoritmo utilizando \textbf{busca em profundidade}. Essa abordagem não se mostrou eficiente, já que movimentava o jogo de acordo com a ordem na qual foi estabelecido os espaços vizinhos, fazendo assim uma série de movimentos desnecessários até a eventual obtenção do resultado.
\\

Em seguida, foi construido um algoritmo usando \textbf{busca em largura}. Esse método de busca se mostrou mais eficaz que o anterior, com caminhos mais diretos até o resultado.

...

\end{document}