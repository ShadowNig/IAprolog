\documentclass[12pt,a4paper]{article}
\usepackage[utf8]{inputenc}
\usepackage[portuguese]{babel}
\usepackage[T1]{fontenc}
\usepackage{amsmath}
\usepackage{amsfonts}
\usepackage{amssymb}
\author{Maqueise Pinheiro}
\title{Deslocamento Universitário: Conclusão}
\date{Maio de 2019}
\begin{document}
\maketitle
O objetivo desse trabalho era analisar se havia alguma relação entre o tempo gasto por estudantes universitários no trânsito e seus respectivos rendimentos escolares. Para isso, foi feita uma análise descritiva, onde foram apresentados alguns gráficos, e um teste de independência para avaliar a relação das variáveis.

Nos primeiros gráficos apresentados foi possível vizualizar a distribuição das idades. A maior parte dos 159 alunos que responderam o questionário têm idades entre 18 e 24 anos, sendo que o mais novo tem 18 e o mais velho 32 anos de idade.

Nos gráficos seguintes, é possível ver a distribuição dos coeficiêntes de rendimento cujo a maior parte dos alunos possuem em média classificação 7.

Foi construído uma tabela para melhor vizualização da variável sobre pegar ou não trânsito na ida e volta da faculdade, e verificou-se que a maioria dos alunos enfrentam trânsito em seus trajetos. Foi visto que 83\% dos alunos acreditam que seus rendimentos escolares  são prejudicados pelo trânsito. 

A maior parte dos alunos gastam de 30 min até 1h por dia em seus trajetos sem o fator trânsito, enquanto em um dia com trânsito a maior parte gasta de 1h a 1:30h.

Por fim, foi realizado um teste de independência equivocadamente sob as variáveis "tempo gasto sem trânsito" e "tempo gasto com trânsito" e, como era previsto, elas possuiam dependência. O que deveria ter sido feito para atender ao objetivo do trabalho, era realizar esse mesmo teste mas em relação as variáveis "tempo gasto com trânsito" e "coeficiente de rendimento". Sendo assim, o trabalho mostrou-se inconclusivo, já que o único teste realizado foi incoerente.
\end{document}